
\documentclass[11pt]{article}
\usepackage{graphicx}
\begin{document}

\section{
Quantum Art Manifesto}




Quantum "computing" is the extension of so-called computer "science" into the realm of quantum information.  The quantum art project is an extension of my rejection of the standard model of a "computer" into the quantum realm.




    I claim that what we call "computers" are primarily art replicators.  By "primarily", I mean that replication of art is the thing that most accurately describes the overall behavior of computer systems.  Most of why computers have changed all aspects of the human condition have to do with sharing art in one way or another, usually graphical.  Just as we view DNA replication as totally fundamental to life, we must examine how machines replicate to understand their most important properties.  Every existing computer came into being because someone made a decision to allocate limited resources to make <em>that</em> machine and not some other competing machine.  That decision was almost always in response to a pitch made by an advocate of that machine using, probably PowerPoint(or maybe Keynote).  While of course the machine used to project the pitch is a "computer", as is the machine built based on that pitch, those are not what determine the overall evolution of machine systems.  The art replication properties very much determine it!  This is why the evolution of the "computer" has moved to more and more screen area and density, with more direct human interaction, and more pictures, rather than just endless supercomputers doing math. 




    Computer "scientists" believe that the "computer" can always be modeled by a imaginary device called a Turing Machine.  They then prove things about this machine, and build them. But nowhere in their model is the technician who maintains the machine, the factory which builds it, the land fill where it ends up, or the investor. who funds it.  These are not trivial oversights, they are fundamental!  Imagine two Turing machines.  One is vastly superior in all technical qualities: speed of operation, density of data tape, etc.  It is perfect in almost every way, but it literally does nothing but carry out bit operations on a binary tape.  The other machine barely works as a computing machine, but creates pretty banners, which can be sold for a very high price as art. Which one will get replicated faster?  Which will end up then being the ancestor of the future, better machines?  I argue that it is always the one that appeals to the human making future decisions about allocation of resources.  Technical people continually are confounded by this and claim it has something to do with "irrational" humans, but I argue that it is in fact because their model of the machine is wrong at a very fundamental level.  THe machine replication is not <em>random</em>, it follows rules, and those rules can be studied--they're just not about computation, they're about how the human mind processes language.  




    But what of quantum computers?  



We live in an age of quantum uncertainty.

\begin{itemize}

    \item
We should stop taking things literally.  From notions about a God-creator to algebraic formulas to models of computation machines and even fundamental laws of the Universe from particle physics we must realize that all of it is filtered through language in a noisy way that literalism hides this noise in a counterproductive way.
    \item
The rise of quantum information science in the physics lab since the 1980s has corresponded with a rise in other failures of binary or otherwise discrete notions of how we choose to parse reality(e.g. the failure of the gender binary model for many people)./
    \item

        Quantum computers are art replicators like any other machine.  Since they are being made, the quantum artists studies how to make art which best aids in machine replication, largely to see if it can be done.  Can building a easily used free language of cartoon graphics for selling quantum computers in funding pitches affect machine replication?  We will study this questions by actually doing this.
    
    \item

        If we stop taking ideas literally, there is much to learn about the direct world around us in the so-called classical regime from quantum information science.  Given this, developing artistic tools in Hilbert space which can allow artists to depict ideas of superposition and entanglement can help people in our society to better understand the uncertain world our minds now inhabit.  
    
\end{itemize}


What all this adds up to is two main branches on the Tree of Quantum Art:

\begin{enumerate}

    \item

        Art that helps quantum machines replicate, to be used by people in the field, designed to integrate into existing workflows for promoting their work and
    
    \item

        Art tools which allow people from the art world with no quantum background to explore the quantum world artistically, creating art that embodies quantum ideas, namely superposition, entanglement, and the geometry of standard Hilbert spaces of various numbers of qubits. 
    
\end{enumerate}
\end{document}
